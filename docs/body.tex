% ----------------------------------------------------------------
\section{Introduction}
\label{sec:introduction}

\input 01intro

\section{Step 1: The Basics}

\input 02step1

\section{Step 2: Blocking}

\input 03step2

\section{Step 3: Blocking for Multiple Levels of Cache}

\input 04step3

\section{Step 4: Parallelizing with OpenMP}

\input 05step4

% ----------------------------------------------------------------


\NoShow{
\section{Organization of the Project}

\input DirectoryTree

One goal of this exercise is to teach the reader how a software library project is often organized into directories.  We acknowledge that the structure is probably overkill for this relatively simple situation, but hope that it has value nonetheless.



% ----------------------------------------------------------------
Architecture: Ivy Bridge and Haswell.

\section{Naive Approach: Three loops}



\section{Cache Blocking: 6 loops}

refer to GOTO paper: How to permuate to get the best loop order
var2, var1, var3


Performance Graph

\section{Add packing}


\section{Kernel Tricks}
1. Butterfly or Broadcasting?

2. Double buffering



\section{Parameter Tuning}




\section{Parallelization}








}

% ----------------------------------------------------------------
\section{Conclusion}
\label{sec:conclusion}

Conclusion.


\subsection*{Useful links}

\noindent
Documentation
\begin{itemize}
	\item 
    \myhref{http://shpc.ices.utexas.edu/}{The Science of High-Performance Computing (SHPC) group website}.
    \item
    \myhref{http://www.cs.utexas.edu/users/flame/web/FLAMEPublications.html}{The FLAME project publications webpage}.  (The umbrella project that BLIS is part of is known as the FLAME project.)
    \item \myhref{https://software.intel.com/sites/landingpage/IntrinsicsGuide/}{Intel Intrinsics Guide}.
    \item \myhref{https://software.intel.com/en-us/isa-extensions}{Intel ISA Extensions}.
    \end{itemize}
    
\noindent
Software
\begin{itemize}
	\item
	\myhref{http://https://github.com/flame/blis}{BLIS on GitHub}.
	\item
	\myhref{https://software.intel.com/en-us/qualify-for-free-software}{Intel Free Software} (including C/C++ compilers).
	\item
	\myhref{https://software.intel.com/en-us/intel-mkl}{Intel's Math Kernels Library (MKL) website}.
	\item
	\myhref{https://software.intel.com/en-us/articles/free_mkl}{Download MKL for free}.
\end{itemize}
