% THEOREMS -------------------------------------------------------

\setlength{\topsep}{0pt}
% \newtheorem{algorithm}{Algorithm}[chapter] % already in SIAM macros
% \newtheorem{lemma}{Lemma}[section] % already in SIAM macros
% \newtheorem{corollary}{Corollary}[chapter] % already in SIAM macros
% \newtheorem{example}{Example} % already in SIAM macros
% \newtheorem{theorem}{Theorem}[chapter] % already in SIAM macros
% \newtheorem{definition}{Definition}[chapter] % already in SIAM macros
% \newtheorem{definition}{Definition} % already in SIAM macros
\newtheorem{remark}{Remark}
\newtheorem{exercise}{Exercise}
% \renewenvironment{proof}{\noindent
% {\bf Proof:}}
% {\hfill {\bf endofproof}}

\newenvironment{boxedexample}{
\noindent
\newline
\begin{boxedminipage}[t]{\textwidth}
\addtocounter{ourexample}{1}%
{\bf Example \theourexample.}
\em
}
{
\end{boxedminipage}
}

\newenvironment{boxedexercise}{
\noindent
\newline
\begin{boxedminipage}[t]{\textwidth}
\addtocounter{exercise}{1}%
{\bf Exercise \theexercise.}
\em
}
{
\end{boxedminipage}
}

\newenvironment{boxit}{
\noindent
\newline
\begin{boxedminipage}[t]{\textwidth}
}
{
\end{boxedminipage}
}

% \newenvironment{boxedremark}{%
% \noindent
% \newline
% \begin{boxedminipage}[t]{\textwidth}
% \begin{remark}
% }
% {
% \end{remark}
% \end{boxedminipage}
% }

\newenvironment{boxedremark}{%
%
\noindent
\begin{boxedminipage}[t]{\textwidth}
\addtocounter{remark}{1}%
{\bf Remark \theremark.}
\em
}
{
\end{boxedminipage}
}


\newenvironment{notetoexpert}{%
\noindent
\newline
\begin{boxedminipage}[t]{\textwidth}
\addtocounter{remark}{1}%
{\bf Note to the skeptical expert.}
\em
}
{
\end{boxedminipage}
}


\newenvironment{boxeddefinition}{%
\noindent
\newline
\begin{boxedminipage}[t]{\textwidth}
\addtocounter{mydefinition}{1}%
{\bf Definition \themydefinition.}
\em
}
{
\end{boxedminipage}
}

\newcommand{\eop}{
{\bf endofproof}
}

\newcommand{\I}{\mathbb{I}}
\renewcommand{\S}{\mathbb{S}}

\newcommand{\Rm}{\R^{m}}
%\newcommand{\Rn}{\R^{n}}
\newcommand{\Rk}{\R^{k}}

\newcommand{\Rmxm}{\R^{m \times m}}
\newcommand{\Rnxn}{\R^{n \times n}}
\newcommand{\Rmxn}{\R^{m \times n}}
\newcommand{\Rnxm}{\R^{n \times m}}

\newcommand{\Rmxk}{\R^{m \times k}}
\newcommand{\Rkxn}{\R^{k \times n}}
\newcommand{\Rnxk}{\R^{n \times k}}
\newcommand{\Rkxk}{\R^{n \times k}}

\newcommand{\Sm}{\S^{m}}
\newcommand{\Sn}{\S^{n}}
\newcommand{\Sk}{\S^{k}}
\newcommand{\Sonexm}{\S^{1 \times m}}

\newcommand{\Smxm}{\S^{m \times m}}
\newcommand{\Snxn}{\S^{n \times n}}
\newcommand{\Smxn}{\S^{m \times n}}
\newcommand{\Snxm}{\S^{n \times m}}

\newcommand{\Smxk}{\S^{m \times k}}
\newcommand{\Skxn}{\S^{k \times n}}
\newcommand{\Snxk}{\S^{n \times k}}

%\newcommand{\alert}{!!!}
\newcommand{\row}[1]{\check{#1}}
\newcommand{\tr}[1]{{#1}^{\mathrm{T}}}
\newcommand{\tc}[1]{{#1}^{\mathrm{H}}}

\newcommand{\TrLw}[1]{\mathrm{TrLw}(#1)}
\newcommand{\TrUp}[1]{\mathrm{TrUp}(#1)}
\newcommand{\SyLw}[1]{\mathrm{SyLw}(#1)}
\newcommand{\SyUp}[1]{\mathrm{SyUp}(#1)}
\newcommand{\HeLw}[1]{\mathrm{HeLw}(#1)}
\newcommand{\HeUp}[1]{\mathrm{HeUp}(#1)}

\newcommand{\defcolvector}[2]{
\left( 
\begin{array}{c}
#1_0 \\ 
#1_1 \\ 
\vdots \\ 
#1_{#2-1}
\end{array} 
\right)
}

\newcommand{\defrowvector}[2]{
\left(#1_0, #1_1, \ldots, #1_{#2-1}\right)
}
%\left( 
%\begin{array}{c c c c}
%#1_0, & #1_1, & \ldots, & #1_{#2-1}
%\end{array} 
%\right)
%}

\newcommand{\defmatrix}[3]{
\left( 
\begin{array}{c c c c}
#1_{00}     & #1_{01}      & \ldots & #1_{0,#3-1} \\ 
#1_{10}     & #1_{11}      & \ldots & #1_{1,#3-1} \\ 
\vdots      & \vdots       & \ddots & \vdots    \\ 
#1_{#2-1,0} & #1_{#2-1,0}  & \ldots & #1_{#2-1,#3-1} \\ 
\end{array} 
\right)
}

\newcommand{\deflowtrmatrix}[3]{
\left( 
\begin{array}{c c c c}
#1_{00}     & 0            & \ldots & 0           \\ 
#1_{10}     & #1_{11}      & \ldots & 0           \\ 
\vdots      & \vdots       & \ddots & \vdots    \\ 
#1_{#2-1,0} & #1_{#2-1,0}  & \ldots & #1_{#2-1,#3-1} \\ 
\end{array} 
\right)
}


\newcommand{\unit}{\mbox{\bf u}}

\newcommand{\sign}{\mbox{\rm sign}}

\newcommand{\Chol}[1]{\mbox{\sc Chol}\!\left( #1 \right)}
\newcommand{\QR}[1]{\mbox{\sc QR}( #1 )}
\newcommand{\LU}[1]{\mbox{\sc LU}( #1 )}
\newcommand{\LUpiv}[1]{\mbox{\sc LUpiv}( #1 )}
\newcommand{\tril}[1]{\mbox{\sc tril}( #1 )}
\newcommand{\triu}[1]{\mbox{\sc triu}\!\left(  #1 \right)}
\newcommand{\trilu}[1]{\mbox{\sc trilu}( #1 )}
\newcommand{\triuu}[1]{\mbox{\sc triuu}( #1 )}
\newcommand{\housev}[1]{\mbox{\sc Housev}( #1 )}

\newcommand{\Scal}{\mbox{\sc Scal}}
\newcommand{\Gemv}{\mbox{\sc Gemv}}
\newcommand{\Symv}{\mbox{\sc Symv}}
\newcommand{\Ger}{\mbox{\sc Ger}}
\newcommand{\Syr}{\mbox{\sc Syr}}
\newcommand{\Trmv}{\mbox{\sc Trmv}}
\newcommand{\Trsv}{\mbox{\sc Trsv}}
\newcommand{\Gemm}{\mbox{\sc Gemm}}
\newcommand{\Gepp}{\mbox{\sc Gepp}}
\newcommand{\Gebp}{\mbox{\sc Gebp}}
\newcommand{\Gepdot}{\mbox{\sc Gepdot}}
\newcommand{\Gepb}{\mbox{\sc Gepb}}
\newcommand{\Gemp}{\mbox{\sc Gemp}}
\newcommand{\Gepm}{\mbox{\sc Gepm}}
\newcommand{\Symm}{\mbox{\sc Symm}}
\newcommand{\Symp}{\mbox{\sc Symp}}
\newcommand{\Sypm}{\mbox{\sc Sypm}}
\newcommand{\Sybp}{\mbox{\sc Sybp}}
\newcommand{\Sypb}{\mbox{\sc Sypb}}
\newcommand{\Sypdot}{\mbox{\sc Sypdot}}
\newcommand{\Syrk}{\mbox{\sc Syrk}}
\newcommand{\Sypp}{\mbox{\sc Sypp}}
\newcommand{\Trmm}{\mbox{\sc Trmm}}
\newcommand{\Trmp}{\mbox{\sc Trmp}}
\newcommand{\Trpm}{\mbox{\sc Trpm}}
\newcommand{\Trbp}{\mbox{\sc Trbp}}
\newcommand{\Trpb}{\mbox{\sc Trpb}}
\newcommand{\Trsm}{\mbox{\sc Trsm}}
\newcommand{\Trsmp}{\mbox{\sc Trsmp}}
\newcommand{\Trspm}{\mbox{\sc Trspm}}
\newcommand{\Trsbp}{\mbox{\sc Trsbp}}
\newcommand{\Trspb}{\mbox{\sc Trspb}}



% Macros added by Jianyu
\def\gap{\hspace*{.2in}}

\newcommand{\sketchplan}[1]{[\emph{#1}]}

% references
\newcommand{\figref}[1]{Figure~\ref{#1}}
\newcommand{\tabref}[1]{Table~\ref{#1}}
\newcommand{\secref}[1]{\S\ref{#1}}
\newcommand{\algref}[1]{Algorithm~\ref{#1}}
\newcommand{\pref}[1]{page~\pageref{#1}}

\newcommand{\todo}[1]{{\color{red}todo: #1}}
\newcommand{\commentblue}[1]{{\color{blue}comment: #1}}
\newcommand{\pkg}[1]{{\color{red} #1}}
\renewcommand{\b}[1]{{#1}}
\newcommand{\grbf}[1]{\mbox{\boldmath${#1}$\unboldmath}} %\renewcommand{\grbf}[1] {\mathbf{#1}}
\newcommand{\p} {\partial}

\newcommand{\Div}{\mbox{div}\,}
\newcommand{\Grad}{{\gbf \nabla}}
\newcommand{\Lap}{\rotatebox[origin=c]{180}{$\nabla$}}
\newcommand{\Curl}{{\gbf \nabla \times}}

\newcommand{\half}[1]{\frac{#1}{2}}
\DeclareMathOperator*{\argmin}{arg\,min}
\newcommand{\MA}[1]{{\mathcal #1}}
\DeclareMathOperator{\bigO}{\mathcal{O}}
\newcommand{\reals}{\mathbb{R}}

\newcommand{\codenm}[1]{{\tt  #1}}
\newcommand{\zapspace}{\topsep=0pt\partopsep=0pt\itemsep=0pt\parskip=0pt}


\newcommand{\msection}[1]{\vspace*{-5pt}\section{#1}\vspace{-4pt}}
\newcommand{\msubsection}[1]{\vspace*{-5pt}\subsection{#1}\vspace{-4pt}}
%\newcommand{\msection}[1]{\section{#1}}
%\newcommand{\msubsection}[1]{\subsection{#1}}


\definecolor{light-gray}{gray}{0.80}
\renewcommand{\algorithmiccomment}[1]{#1}

%\newcommand\mycommfont[1]{\footnotesize\ttfamily\textcolor{blue}{#1}}
%\SetCommentSty{mycommfont}


%\newcommand{\algcolor}[2]{\hspace*{-\fboxsep}\colorbox{#1}{\parbox{\linewidth}{#2}}}
\newcommand{\algcolor}[2]{\colorbox{#1}{\parbox{\linewidth}{#2}}}
\newcommand{\algemph}[1]{\algcolor{light-gray}{#1}}

\newcommand{\algcmt}[1]{\hfill {\footnotesize\ttfamily\textcolor{blue}{/* {#1} */}}}
%\newcommand{\algcmt}[1]{{\footnotesize\ttfamily\textcolor{blue}{\dotfill #1}}}
%\newcommand{\algcc}[1]{\hfill {\footnotesize\ttfamily\textcolor{blue}{//{#1}}}}
\newcommand{\algcc}[1]{ {\tiny\ttfamily\textcolor{blue}{//{#1}}}}
\newcommand{\algrcc}[1]{\hfill {\tiny\ttfamily\textcolor{blue}{//{#1}}}}




\newcommand{\norm}[1]{\left\lVert #1 \right\rVert}

%----------------------------------------------------------------------------------------
%	CODE INCLUSION CONFIGURATION
%----------------------------------------------------------------------------------------

%\definecolor{MyDarkGreen}{rgb}{0.0,0.4,0.0} % This is the color used for comments
%\lstloadlanguages{C++} % Load Perl syntax for listings, for a list of other languages supported see: ftp://ftp.tex.ac.uk/tex-archive/macros/latex/contrib/listings/listings.pdf
%\lstset{language=Matlab, % Use Perl in this example
%        frame=single, % Single frame around code
%        basicstyle=\small\ttfamily, % Use small true type font
%        keywordstyle=[1]\color{Blue}\bf, % Perl functions bold and blue
%        %keywordstyle=[1]\color{Purple}\bf, % Perl functions bold and blue
%        keywordstyle=[2]\color{Purple}, % Perl function arguments purple
%        keywordstyle=[3]\color{Blue}\underbar, % Custom functions underlined and blue
%        identifierstyle=, % Nothing special about identifiers                                         
%        commentstyle=\usefont{T1}{pcr}{m}{sl}\color{MyDarkGreen}\small, % Comments small dark green courier font
%        stringstyle=\color{Purple}, % Strings are purple
%        showstringspaces=false, % Don't put marks in string spaces
%        tabsize=4, % 5 spaces per tab
%        %
%        % Put standard Perl functions not included in the default language here
%        morekeywords={rand, Matrix, Vector},
%        %
%        % Put Perl function parameters here
%        morekeywords=[2]{bool, int, double, char, template},
%        %
%        % Put user defined functions here
%        morekeywords=[3]{T},
%       	%
%        morecomment=[l][\color{Blue}]{...}, % Line continuation (...) like blue comment
%        numbers=left, % Line numbers on left
%        firstnumber=1, % Line numbers start with line 1
%        numberstyle=\tiny\color{Blue}, % Line numbers are blue and small
%        stepnumber=1 % Line numbers go in steps of 5
%}
%
%% Creates a new command to include a perl script, the first parameter is the filename of the script (without .pl), the second parameter is the caption
%\newcommand{\perlscript}[2]{
%  \begin{itemize}
%	\item[]\lstinputlisting[caption=#2,label=#1]{#1.cpp}
%  \end{itemize}
%}

\usepackage{pifont}% http://ctan.org/pkg/pifont
\newcommand{\cmark}{\ding{51}}%
\newcommand{\xmark}{\ding{55}}%
\newcommand{\myhref}[2]{\href{#1}{\ding{42} #2}}
